\newcommand{\pluginName}{Heston Model}
\newcommand{\pluginVersion}{1.0.9}

\input{../../../DocumentationTemplate/TemplateL3}

\begin{document}

\PluginTitle{\pluginName}{\pluginVersion}

\section{Introduction}
This plug-in implements the Heston model. Once installed the plug-in offers the possibility of using two new processes, the Heston process and the Heston time dependent drift process. 
The latter is a generalization of the former in which the drift is time-dependent. For a general reference on the Heston model see \cite{Heston:ClosedFormSol}.

\section{How to use the plug-in}

\subsection{Heston process}
in the fairmat user interface, while adding a new stochastic process, Heston model can be selected.
The stochastic process is defined by the parameters shown in the table below:
\begin{center}
\begin{tabular}{|l|c|}
  \hline
\textbf{Fairmat}&\textbf{Documentation}\\
\textbf{notation}&\textbf{notation}\\
                     \hline
 S0     & $S_0$\\
 V0     & $V_0$\\
 r  & $r$\\
 q  & $q$\\
 k      & $k$ \\
 theta  & $\theta$\\
 sigma  & $\sigma$\\
rho & $\rho$ \\ 
   \hline
\end{tabular}
\end{center}
$S_0$ and $V_0$ are the starting values for the stock process and the volatility process, the remaining parameters regulate the model dynamic.

It is possible to edit the model values shown in the table above directly on the ``Parameters'' tab of the Heston process window by double clicking on them. To set the value of the correlation parameter (called $\rho$ in the following) you have to select the Heston process in the ``Stochastic Process'' window and then click on ``Correlation''.

\subsection{Heston with time dependent drift}
in the fairmat user interface, while adding a new stochastic process, Heston model with time dependent drift can be selected.
The stochastic process is defined by the parameters shown in the table below:
\begin{center}
\begin{tabular}{|l|c|}
  \hline
\textbf{Fairmat}&\textbf{Documentation}\\
\textbf{notation}&\textbf{notation}\\
                     \hline
 S0     & $S_0$\\
 V0     & $V_0$\\
 k      & $k$ \\
 theta  & $\theta$\\
 sigma  & $\sigma$\\
 zero rate curve    & $\mathrm{ZR}(t)$\\
 dividend yield curve    & $q(t)$\\
rho & $\rho$ \\ 
   \hline
\end{tabular}
\end{center}
$S_0$ and $V_0$ are the starting values for the stock process and the volatility process, the remaining parameters regulate the model dynamic. They are all scalar except the Drift Curve, for this parameter you have to insert a reference to a curve.

It is possible to edit the model values as in the Heston case. Remember that to specify a reference to a previous defined curve {\ttfamily mu} you have to use the notation {\ttfamily @mu}.

\section{Implementation Details}

The Heston model is used to describe the evolution of a stock price (or an index) with stochastic volatility. It is defined by the following stochastic differential equations
\begin{align}
& dS(t) = \mu S(t)dt + \sqrt{V(t)}S(t)dW_1(t)\label{eq:dsh}\\
& dV(t) = k(\theta - V(t))dt + \sigma\sqrt{V(t)}dW_2(t)\\
& \mathbb{E}[dW_1(t)dW_2(t)] = \rho dt
\end{align}
where $S$ represents the price process, $V$ represents the volatility process and $W_1, W_2$ are correlated Wiener processes with instantaneous correlation $\rho$.

The price process follows a geometric Brownian motion with a stochastic volatility while the volatility follows a square root mean reverting process $($Cox$-$Ingersoll$-$Ross model$)$. Usually $\rho$ is negative, so a decrease in the stock level corresponds to a volatility increment. 

The parameters have the following interpretation:
\begin{itemize}
\item $\mu$ is the rate of return of the stock price
\item $k$ is the speed of mean reversion
\item $\theta$ is the mean reversion level. As $t$ goes to infinity, the expected value of $V(t)$ tends to $\theta$:
\item $\sigma$ is the "volatility of volatility", in other words it regulates the variance of $V(t)$.
\end{itemize}

Parameters $k$, $\theta$ and $\sigma$ have to satisfy the constrain $2k\theta>\sigma^2$ (known as the Feller condition) in order to exclude the possibility for $V(t)$ to reach 0.

In the case of Heston with constant drift $\mu$ is set equal to $r-q$ where $r$ is the risk free rate and $q$ is the dividend yield rate of the stock. 
Please note that a constant discount factor should be selected while using the constant risk free rate $r$.

The Heston model with time dependent drift is defined by the same stochastic differential equations with the only difference that the $\mu$ parameter is time-dependent so that equation (\ref{eq:dsh}) becomes
\begin{equation}
dS(t) = \mu(t) S(t)dt + \sqrt{V(t)}S(t)dW_1(t).
\end{equation}
In this case the function $\mu(t)$ is given by
\begin{equation}\label{eq:mu}
\mu(t) = \frac{d\mathrm{ZR}(t)}{dt}t + \mathrm{ZR}(t) - q(t)
\end{equation}
Indeed if we have to fix a time dependent deterministic short rate $r(t)$ coherent with an observed zero rate function $\mathrm{ZR}(t)$ we have to impose that prices of zero coupon bond are given by both expression
\begin{equation}
P(0,t) = e^{-\int_0^t r(s)ds} = e^{-\mathrm{ZR}(t)t}
\end{equation}
equating the two exponents we have
\begin{equation}
\int_0^t r(s)ds = \mathrm{ZR}(t)t
\end{equation}
and deriving this expression in $t$ gives us
\begin{equation}
r(t) = \frac{d\mathrm{ZR}(t)}{dt}t + \mathrm{ZR}(t).
\end{equation}
Given that $\mu(t) = r(t) - q(t)$, Equation \ref{eq:mu} has been proven.


\subsection{Simulation and discretization scheme}

By applying straight Euler-Maruyama method to simulate a Heston process, we obtain this formula for the volatility component
\begin{equation}
V_{t+1} = V_t + k(\theta - V_t)\Delta t + \sigma\sqrt{V_t}\sqrt{\Delta t} N(0,1)
\end{equation}
where $\Delta t = t_{n+1}-t_n$ and $N(0,1)$ represents a realization of a standard Normal random variable. Even if the parameters satisfy the Feller condition it is possible for the discretized version of $V$ to reach negative values.

This forces to use a different discretization scheme. In Fairmat simulations are carried out through Euler full truncation method described in \cite{HaastrechtPelsser:EffHestonSim}, characterized by the equations
\begin{align}
&\log(S_{t+1}) = \log(S_t) + \left(\mu_t - V_t^+/2\right)\Delta t + \sqrt{V_t^+\Delta t} N_1(0,1)\\
&V_{t+1} = V_t + k\left(\theta - V_t^+\right)\Delta t + \sigma\sqrt{V_t^+\Delta t} N_2(0,1)\label{eq:Vdisc}
\end{align}
where we have used the notation $V_t^+=\max\{V_t,0\}$ and where $N_1(0,1), N_2(0,1)$ are realizations of two $\rho$-correlated standard Normal random variable.

Equation (\ref{eq:Vdisc}) can still generate negative values for $V$ but when this happens the next simulation step will have the diffusion term suppressed letting the drift term take the process toward positive values.

This kind of scheme entails no discretization error in the stock price process simulation (with the assumption that $V_t$ remains constant in every $\Delta t$) while it introduces a small bias in the volatility process. To reduce bias error it is appropriate to simulate using small time steps, for example in \cite{HaastrechtPelsser:EffHestonSim} it is suggested to use at least 32 time steps per year. Both in \cite{HaastrechtPelsser:EffHestonSim} and in \cite{Lord:CompBiasSimSchemes} it is stated that this discretization scheme, compared with other Euler-like schemes, seems to produce the smallest bias.

This algorithm is used for both Heston and Heston time dependent drift processes.


\section{Pricing of European options}

\subsection{Characteristic function }

The characteristic function is defined as
\begin{equation}
\phi(u,t) = \mathbb{E} \left[\left.e^{iu\log(S(t))}\right|S_0,V_0\right].
\end{equation}
For the Heston model it is possible to obtain an explicit formula for the characteristic function
\begin{align}
\phi(u,t) = \exp( & iu(\log S_0 + \mu t)) \nonumber\\
 & \cdot\exp(\theta k \sigma^{-2}((k-\rho\sigma iu - d)t - 2\log( (1-ge^{-dt})/ (1-g) )))\nonumber\\
 & \cdot\exp(V_0\sigma^{-2}(k-\rho\sigma iu - d)(1-e^{-dt})/(1-ge^{-dt}))
\end{align}
where
\begin{align}
 d & = \sqrt{(\rho\sigma iu - k)^2 + \sigma^2(iu + u^2)} \\
 g & = \frac{k-\rho\sigma iu - d}{k-\rho\sigma iu + d}.
\end{align}
The form of the $\phi$ function is not equal to that given in Heston original paper but it is an equivalent one which does not have the continuity problem when integrated to find the price of an European call option. This issue is described in \cite{Albrecher:HestonTrap}.




\subsection{Call price formula}
\label{call}

The price of an European call option with strike $K$ and time to maturity $T$ is given by the formula

\begin{equation}
\begin{aligned}
C(K,T, S_0, V_0) & = DF(0,T) \left(	\mathbb{E}\left[S_T 1_{S_T \geq K} \right] - K \mathbb{E}\left[1_{S_T \geq K} \right]\right) \\
&= DF(0,T) \left[ \frac{1}{2}S_0e^{(r-q) T}+ \frac{1}{\pi}\int_0^{\infty}f_1(u) du  \right] \\
&- DF(0,T) K   \left[ \frac{1}{2} + \frac{1}{\pi}\int_0^{\infty}f_2(u) du \right]  \\
&=  \frac{DF(0,T)}{2}\left[S_0e^{(r-q) T} - K\right] + \frac{DF(0,T)}{\pi}\int_0^{\infty}\left[f_1(u) - Kf_2(u)\right]du
\end{aligned}
\end{equation}
where the two functions $f_1$ and $f_2$ are
\begin{align}
f_1(u) & = \mathrm{Re}\left[\frac{\exp(-iu\log K)\phi(u-i,T)}{iu}\right]\\
f_2(u) & = \mathrm{Re}\left[\frac{\exp(-iu\log K)\phi(u,T)}{iu}\right],
\end{align}
and where $DF(t,T)$ represents the discounting factor between time $t$ and $T$. 
In the case of a flat discounting curve with discounting rate equal to $r$ we have the following
$$
DF(t,T) = e^{-r(T-t)} \ .
$$


\subsection{Digital call price formula}
\label{digital}
The price of an European digital call option with strike $K$ and time to maturity $T$ is given by the formula
\begin{equation}
\begin{aligned}
DC(K,T, S_0, V_0) &=DF(0,T) \mathbb{P}(S_T > K) \\ &=  DF(0,T)\mathbb{E}\left[1_{S_T > K}\right] \\
&= DF(0,T) \left[ \frac{1}{2} + \frac{1}{\pi}\int_0^{\infty}f_2(u) du \right] 
\end{aligned}
\end{equation}

Similarly, the price of an European digital put option with strike $K$ and time to maturity $T$ is given by the formula
\begin{equation}
\begin{aligned}
DP(K,T, S_0, V_0) &= DF(0,T) P(S_T \leq K) \\ 
&= DF(0,T) (1-P(S_T > K))  \\ 
& = DF(0,T)  \left[  \frac{1}{2} - \frac{1}{\pi}\int_0^{\infty}f_2(u) du \right] 
\end{aligned} 
\end{equation}


\section{Pricing of forward starting options}
In this section we describe how the pricing of forward starting options can be done in the context of Heston's model. 

\subsection{Ahlpi and Rutkowski approach}
The approach undertaken to price forward starting options is taken from \cite{ahlip2009forward}.


\subsection{Approximated approach}
To describe the approximate approach, let us consider a forward starting call options, with maturity $T$, strike date $T_0$, and strike percentage $\overline{K}$. 
The pricing problem, at time $t$, consists in determining the following expectation
$$
FS(\overline{K}, T, T_0, S_0, V_0)  = DF(t,T) \mathbb{E}_t   \left[\left(S_T - \overline{K} S_{T_0}\right)^+\right] \ .
$$
In the case of $t \in [T_0, T)$ this option corresponds to a vanilla call options and it can be priced using the formula derived in \ref{call}. 
In the case of $t \in [0, T_0)$ we can use an approximated approach taken from the Black-Scholes framework. 
Indeed, in the Black-Scholes framework the price of a forward starting call options can be done as follows:
$$
\begin{aligned}
FS^{BS}(\overline{K}, T, T_0, S_0, V_0) 
&= DF(t,T_0)  \mathbb{E}_t   \left[ DF(T_0,T) S_{T_0}  \mathbb{E}_{T_0} \left[ \left( \frac{S_T}{S_{T_0}} - \overline{K}\right)^+\right]       \right]
\\
&=
DF(t,T_0) \mathbb{E}_t \left[ S_{T_0}  C^{BS}(\overline{K}, T-T_0, 1 ,\sigma) \right] 
\\ &=DF(t,T_0) S_0 e^{(r-q) (T_0-t)}   C^{BS}(\overline{K}, T-T_0, 1 ,\sigma) \ , 
\end{aligned}
$$
where $C^{BS}(\overline{K}, T-T_0, 1 , \sigma)$ is the price of a call option under the Black-Scholes framework, with initial underlying value equal to $1$, volatility parameter $\sigma$, strike $\overline{K}$ and time to maturity $T-T_0$.
In the context of the Heston framework, it is possible to undertake the following approximation
$$
\begin{aligned}
FS(\overline{K}, T, T_0, S_0, V_0) = 
&= DF(t,T_0) \mathbb{E}_t   \left[ DF(T_0,T) S_{T_0}   \mathbb{E}_{T_0} \left[ \left( \frac{S_T}{S_{T_0}} - \overline{K}\right)^+\right]       \right]
\\
&\approx DF(t,T_0)  \mathbb{E}_t   \left[  S_{T_0}  \right] \times \mathbb{E}_t \left[ DF(T_0,T) \mathbb{E}_{T_0} \left[ \left( \frac{S_T}{S_{T_0}} - \overline{K}\right)^+\right]       \right]
\\ 
&=
DF(t,T_0)  S_0 e^{(r-q) (T_0-t)}   C(\overline{K}, T-T_0, 1, \tilde{V}) \ , 
\end{aligned}
$$
with
\begin{equation}
\label{vol-cir-exp}
\tilde{V} = \mathbb{E}_t[ V_{T_0} ] = V_t  e^{-\kappa  (T_0 -t) } + \theta \left(1 - e^{-\kappa   (T_0 -t) }\right) \ ,
\end{equation}

It should be stressed that this approach represents an approximation based on the assumption of absence of correlation between the stock process and the variance process. 









\section{Calibration}

\subsection{Dividend yield calculation}

To estimate the function $\mu(t)$ it is necessary to have the zero coupon curve and a curve describing the expected dividend yield at future dates. The zero coupon curve can be calculated from cash rates and swap rates observed in the market and will be denoted with $ZR(t)$. Expected dividend yields can be calculated through put-call parity relation. In the case of continuous constant dividend yield this relation states that
\begin{equation}
P(K, T, S_0, V_0) = C(K, T, S_0, V_0)  - DF(0,T) \left[ S_0 e^{(r-q)T} - K \right]
\end{equation}
Solving for $q$ we have
\begin{equation}
q = -\frac{1}{T}\ln\left[\frac{C(K, T, S_0, V_0)-P(K, T, S_0, V_0) +DF(0,T) K}{DF(0,T)S_0 e^{rT}}\right].
\end{equation}
Observing call and put prices for at the money options at different maturities we can calculate $q(t)$ and use this values in formula (\ref{eq:mu}).

\subsection{Calibration and Objective function}

Given a stock (or index) with value $S_0$ at a certain date, the price of a call option with strike $K$ and maturity $T$ priced with the Heston model can be seen as a function of the model $C(K,T,S_0, V_0) = f_H(V_0, \mu(T), k, \theta, \sigma, K, T, S_0)$.

Given a matrix of call prices taken from the market $C_M(i,j)$, where the indexes represent different maturities $T_i$ and strike $K_j$, Fairmat fixes the parameters $(V_0, k, \theta, \sigma)$ searching the minimum of the function
\begin{equation}
f(V_0, k, \theta, \sigma) = \sum_{ij}\Big[f_H(V_0, \mu(T_i), k, \theta, \sigma, K_j, T_i, S_0) - C_M(i,j)\Big]^2.
\end{equation}

Information necessary for the calibration is taken from two different data types, one containing an InterestRateMarketData  structure with data needed to calculate the function $ZR(t)$ and one containing a CallPriceMarketData structure with data regarding option prices for the index.

To calibrate the constant drift version you need the same market data and you also have to specify a maturity on the calibration settings. If the maturity value is $T$ then $r$ and $q$ are set to $r = ZR(T)$ and $q = q(T)$. The remaining parameters are found as before but calibration is performed with $\mu(t)$ constant and equal to $r-q$.




\section{Greeks}
In this section, we describe how the calculation of the Greeks can be computed in the framework of the Heston model. 

%%%%%%%%%%%%%%%%%% CALL OPTION %%%%%%%%%%%%%%%%%%%%%%%%%%%%%%

\subsubsection{Greeks of a European call option}
As described in section \ref{call}, the price of a European Call option under the Heston stochastic process is given by the following formula:

$$
C(K,T, S_0, V_0) = 
%\frac{1}{2}\left[S_0e^{-q T} - Ke^{-r T}\right] + \frac{e^{-r T}}{\pi}\int_0^{\infty}\left[f_1(u) - Kf_2(u)\right]du
\frac{DF(0,T)}{2}\left[S_0e^{(r-q) T} - K\right] + \frac{DF(0,T)}{\pi}\int_0^{\infty}\left[f_1(u) - Kf_2(u)\right]du
$$


\begin{itemize}

\item \textbf{Delta} 

The Delta of a call option measures the rate of change of the option price with respect to changes in the underlying asset's price. For a call option, it can be calculated as follows:

$$
\begin{aligned}
\Delta_C(K,T, S_0, V_0) = \frac{ \partial C(K,T, S_0, V_0) }{\partial S_0} &= 
 \frac{DF(0,T)}{2}e^{(r-q) T} \\ &+ \frac{DF(0,T)}{\pi} \int_0^{\infty}\frac{ \partial  }{\partial S_0}  \left[f_1(u) - Kf_2(u)\right]du
\end{aligned}
$$

The derivatives of the functions $f_1(u)$ and $f_2(u)$ with respect to $S_0$ are given by:
\begin{equation}
\label{eq:derivative-s0-f1-f2}
\begin{aligned}
\frac{ \partial  f_1(u)   }{\partial S_0}&= \mathrm{Re}\left[  \frac{\exp(-iu\log K) \phi_{S_0}(u-i,T)   }{iu}\right]   \\  
\frac{\partial f_2(u)   }{\partial S_0} &= \mathrm{Re}\left[ \frac{\exp(-iu\log K) \phi_{S_0}(u,T)  }{iu}\right].
\end{aligned}
\end{equation}

Here, $\phi_{S_0}(u,T)$ is the derivative of the characteristic function $\phi(u,T)$ with respect to $S_0$, and is given by:
$$
\phi_{S_0}(u,T) = \frac{ \partial  \phi(u,T)}{\partial S_0} = \frac{iu}{S_0} \phi(u,T) \ . 
$$


\item \textbf{Gamma} 

The Gamma of a call option is the rate of change in the Delta with respect to changes in the underlying price.  For a call option, it can be calculated as follows:
$$
\begin{aligned}
\Gamma_C(K,T, S_0, V_0) =
\frac{ \partial^2 C(K,T, S_0, V_0) }{\partial S_0^2} &=
 \frac{DF(0,T)}{\pi} \int_0^{\infty}\frac{ \partial^2  }{\partial S_0^2}  \left[f_1(u) - Kf_2(u)\right]du \ .
\end{aligned}
$$
The second derivatives of the functions $ f_1(u) $ and $ f_2(u) $ with respect to $ S_0 $ are given by:
\begin{equation}
\label{eq:s-derivative-s0-f1-f2}
\begin{aligned}
\frac{ \partial^2  f_1(u)   }{\partial S_0^2}&= \mathrm{Re}\left[  \frac{\exp(-iu\log K) \phi_{S_0^2}(u-i,T)   }{iu}\right]  \ , \\  
\frac{\partial^2  f_2(u)   }{\partial S_0^2 } &= \mathrm{Re}\left[ \frac{\exp(-iu\log K) \phi_{S_0^2}(u,T)  }{iu}\right].
\end{aligned}
\end{equation}
Here, $\phi_{S_0^2}(u,T)$ is the second derivative of the characteristic function $\phi(u,T)$ with respect to $S_0$, and is given by:
$$
\begin{aligned}
\phi_{S_0^2}(u,T) = \frac{ \partial^2  \phi(u,T)}{\partial S_0^2} &= \frac{\partial}{\partial S_0}  \left[ \frac{iu}{S_0} \phi(u,T)\right] \\ &= -\frac{iu}{S_0^2} \phi(u,T) + \left( \frac{iu}{S_0} \right)^2 \phi(u,T) \ .
\end{aligned}
$$




\item \textbf{Rho}

The Rho of a call option measures the rate of change of the option price with respect to changes in the risk-free interest rate. It can be represented as follows:

%$$
%\begin{aligned}
%\text{Rho}_C(K,T, S_0, V_0) &= 
%\frac{ \partial C(K,T, S_0, V_0) }{\partial r} 
%\\ &=
%\frac{T}{2} Ke^{-r T}  - \frac{T e^{-r T}} {\pi}\int_0^{\infty}\left[f_1(u) - Kf_2(u)\right]du 
%\\&+ \frac{e^{-r T}} {\pi}\int_0^{\infty}\frac{ \partial \left[f_1(u) - Kf_2(u)\right]}{\partial r}   du \ .
%\end{aligned}
%$$
% \frac{DF(0,T)}{2}\left[S_0e^{(r-q) T} - K\right] + \frac{DF(0,T)}{\pi}\int_0^{\infty}\left[f_1(u) - Kf_2(u)\right]du
$$
\begin{aligned}
\text{Rho}_C(K,T, S_0, V_0) &= 
\frac{ \partial C(K,T, S_0, V_0) }{\partial r} 
\\ &=
\frac{ \partial DF(0,T) }{\partial r} \left[ \frac{1}{2}\left[S_0e^{(r-q) T} - K\right] + \frac{1}{\pi}\int_0^{\infty}\left[f_1(u) - Kf_2(u)\right]du \right]
\\
&+
DF(0,T)\left[ \frac{T}{2} S_0 e^{(r-q) T} +\frac{1} {\pi}\int_0^{\infty}\frac{ \partial \left[f_1(u) - Kf_2(u)\right]}{\partial r}   du \right]
\end{aligned}
$$


The derivatives of the functions $ f_1(u) $ and $ f_2(u) $ with respect to $ r $ are given by:

\begin{equation}
\label{eq:derivative-r-f1-f2}
\begin{aligned}
\frac{ \partial  f_1(u)   }{\partial r}&= \mathrm{Re}\left[  \frac{\exp(-iu\log K) \phi_{r}(u-i,T)   }{iu}\right] \ ,  \\  
\frac{\partial  f_2(u)   }{\partial r } &= \mathrm{Re}\left[ \frac{\exp(-iu\log K) \phi_{r}(u,T)  }{iu}\right].
\end{aligned}
\end{equation}
Here, $\phi_{r}(u,T)$ is the derivative of the characteristic function $\phi(u,T)$ with respect to $r$, and is given by:
$$
\phi_{r}(u,T) = \frac{ \partial  \phi(u,T)}{\partial r} = (i u T) \phi(u,T) \ .
$$




\item \textbf{Theta} 

The Theta of a call option measures the rate of change of the option price with respect to changes in time, or the time decay of the option. It can be represented as follows:
$$
\Theta_C(K,T, S_0, V_0) = - \frac{\partial C(K,T, S_0, V_0)}{\partial T} \ . 
$$

Given the complexity of the Heston model, an analytical solution for the Theta could be complex. Therefore, we use the central difference method, which approximates the derivative of a function at a point.
$$
\Theta_C(K,T, S_0, V_0) \approx  \frac{C(K,T-h) - C(K,T+h)}{2h} \ , 
$$
where $h$ is a small increment in time. 

\item \textbf{Vega} 

In the Heston model, Vega measures the rate of change of the option price with respect to changes in the initial value of the variance $V_0$. It can be derived as follows:

$$
\text{Vega}_C(K,T, S_0, V_0) = \frac{\partial C(K,T, S_0, V_0)}{\partial V_0} = \frac{DF(0,T)}{\pi} \int_0^{\infty}\frac{ \partial  }{\partial V_0}  \left[f_1(u) - Kf_2(u)\right]du \ . 
$$

The derivatives of the functions $ f_1(u) $ and $ f_2(u) $ with respect to $ V_0$ are given by:
\begin{equation}
\label{eq:derivative-v0-f1-f2}
\begin{aligned}
\frac{ \partial  f_1(u)   }{\partial V_0}&= \mathrm{Re}\left[  \frac{\exp(-iu\log K) \phi_{V_0}(u-i,T)   }{iu}\right]  \ , \\  
\frac{\partial  f_2(u)   }{\partial V_0 } &= \mathrm{Re}\left[ \frac{\exp(-iu\log K) \phi_{V_0}(u,T)  }{iu}\right].
\end{aligned}
\end{equation}

Here, $\phi_{V_0}(u,T)$ is the derivative of the characteristic function $\phi(u,T)$ with respect to $V_0$, and is given by:
$$
\begin{aligned}
\phi_{V_0}(u,T) &= \frac{ \partial  \phi(u,T)}{\partial V_0} \\ &= \left[ \sigma^{-2}(k-\rho\sigma iu - d)(1-e^{-dt})/(1-ge^{-dt}) \right] \phi(u,T) \ . 
\end{aligned}
$$

\end{itemize}




%%%%%%% PUT OPTION %%%%%%%%%%%


\subsubsection{Greeks of a European put option}

The Greeks of a European put option can be derived from the Greeks of a European call option using the put-call parity. The put-call parity is a principle that defines a relationship between the price of European put options and calls options of the same class (that is, with the same underlying asset, strike price and expiration date). It is given by:
$$
C(K, T, S_0, V_0) - P(K, T, S_0, V_0)  =DF(0,T)\left[ S_0 e^{(r-q) T} - K \right] \ .
$$
Rearranging the terms, we get the price of a European put option:

$$
P(K, T, S_0, V_0) = C(K, T, S_0, V_0) - DF(0,T)\left[ S_0 e^{(r-q) T} - K \right] \ .
$$

Now, let's calculate the Greeks:

\begin{itemize}

\item \textbf{Delta} 

The Delta of a put option is the rate of change of the option price with respect to changes in the underlying asset's price. It can be calculated from the Delta of a call option as follows:

$$
\Delta_P(K,T,S_0,V_0) = \Delta_C(K,T, S_0, V_0) - DF(0,T) e^{(r-q) T} \ .
$$


\item \textbf{Gamma} 

The Gamma of a put option is the rate of change in the Delta with respect to changes in the underlying price. This is essentially measuring the second-order sensitivity of the option value to price changes in the underlying asset. For a put option, the Gamma is the same as that of a call option:

$$
\Gamma_P(K,T,S_0,V_0) = \Gamma_C(K,T, S_0, V_0) \ .
$$



\item \textbf{Rho} 

The Rho of a put option measures the rate of change of the option price with respect to changes in the risk-free interest rate. It can be represented as follows:

$$
\begin{aligned}
\text{Rho}_P(K,T,S_0,V_0) &= \text{Rho}_C(K,T, S_0, V_0) - \frac{\partial DF(0,T)}{\partial r}\left[ S_0 e^{(r-q) T} - K \right] 
\\
&-
DF(0,T) T S_0 e^{(r-q) T} 
\end{aligned}
$$



\item \textbf{Theta} 

The Theta of a put option measures the rate of change of the option price with respect to changes in time, or the time decay of the option. Similary to the call option case, we approximate it using the central difference method. 
$$
 \Theta_P(K,T,S_0,V_0) \approx \frac{P(K,T-h) - P(K,T+h)}{2h} \ ,
$$
where $h$ is a small increment in time. 


\item \textbf{Vega} 

In the Heston model, Vega measures the rate of change of the option price with respect to changes in the initial value of the variance $V_0$. For a European put option it can be derived using the put-call parity:
$$
\text{Vega}_P(K,T,S_0,V_0) = \text{Vega}_C(K,T, S_0, V_0) \ .
$$

\end{itemize}









%%%%%%% DIGITAL CALL OPTION %%%%%%%%%%%




\subsubsection{Greeks of a European digital call option}
As described in section \ref{digital} the price of a European digital call option under the Heston stochastic process is given by
$$
DC(K,T, S_0, V_0) = DF(0,T)  \left[ \frac{1}{2} + \frac{1}{\pi}\int_0^{\infty}f_2(u) du \right]  \ .
$$


\begin{itemize}
\item \textbf{Delta} 

The Delta of a digital call option measures the rate of change of the option price with respect to changes in the underlying asset's price. For the digital call option, it can be calculated as follows:
$$
\Delta_{DC}(K,T,S_0,V_0) =  DF(0,T)    \frac{1}{\pi}\int_0^{\infty}  \frac{\partial f_2(u)}{\partial S_0}  du  \ , 
$$
where $\frac{\partial f_2(u)}{\partial S_0}$ is derived in Equation \ref{eq:derivative-s0-f1-f2}.


\item \textbf{Gamma} 

The Gamma of an option measures the rate of change in the Delta with respect to changes in the underlying price. This is essentially measuring the second-order sensitivity of the option value to price changes in the underlying asset. For the digital call option, it can be calculated as follows:

$$
\Gamma_{DC}(K,T,S_0,V_0) =  DF(0,T)    \frac{1}{\pi}\int_0^{\infty}  \frac{\partial^2 f_2(u)}{\partial^2 S_0}  du \ , 
$$
where $\frac{\partial^2 f_2(u)}{\partial S_0^2}$ is derived in Equation \ref{eq:s-derivative-s0-f1-f2}.


\item \textbf{Rho} 

The Rho of a digital call option measures the rate of change of the option price with respect to changes in the risk-free interest rate. It can be represented as follows:


%$$
%\begin{aligned}
%\text{Rho}_{DC}(K,T,S_0,V_0) &= 
%\frac{ \partial DC(K,T, S_0, V_0) }{\partial r} 
%\\ &=
%-T * DC(K,T, S_0, V_0) + e^{-r T}  \left[ \frac{1}{2} + \frac{1}{\pi}\int_0^{\infty} \frac{ \partial f_2(u)}{\partial r}  du \right] \ ,
%\end{aligned}
%$$


$$
\begin{aligned}
\text{Rho}_{DC}(K,T,S_0,V_0) &= 
\frac{ \partial DC(K,T, S_0, V_0) }{\partial r} 
\\ &=
\frac{\partial DF(0,T)}{\partial r}  \left[ \frac{1}{2} + \frac{1}{\pi}\int_0^{\infty}f_2(u) du \right]  
\\
&+
 DF(0,T)  \left[\frac{1}{\pi}\int_0^{\infty} \frac{ \partial f_2(u)}{\partial r}  du \right] \ .
\end{aligned}
$$
where $\frac{\partial f_2(u)}{\partial r}$ is derived in Equation \ref{eq:derivative-r-f1-f2}.




\item \textbf{Theta} 

The Theta of a digital call option measures the rate of change of the option price with respect to changes in time, or the time decay of the option. Similary to the call option case, we approximate it using the central difference method. 
$$
 \Theta_{DC}(K,T,S_0,V_0) \approx \frac{DC(K,T-h) - DC(K,T+h)}{2h} \  ,
$$
where $h$ is a small increment in time. 


\item \textbf{Vega} 

In the Heston model, Vega measures the rate of change of the option price with respect to changes in the initial value of the variance $V_0$. For a digital call option, it can be derived as follows:

$$
\text{Vega}_{DC}(K,T,S_0,V_0) = \frac{\partial DC(K,T, S_0, V_0)}{\partial V_0} =  \frac{DF(0,T)  }{\pi}\int_0^{\infty}\frac{\partial f_2(u)}{\partial V_0} du \ ,
$$
where $\frac{\partial f_2(u)}{\partial V_0}$ is derived in Equation \ref{eq:derivative-v0-f1-f2}.


\end{itemize}










%%%%%%% DIGITAL PUT OPTION %%%%%%%%%%%



\subsubsection{Greeks of a European digital put option}

The price of a European digital put option ucan be derived from the price of a European digital call option using the fact that the sum of the prices of a digital call and a digital put is equal to $e^{-rT}$. This gives us:
$$
DP(K,T, S_0, V_0) = DF(0,T) - DC(K,T,  S_0, V_0) \ .
$$
Now, let's calculate the Greeks:

\begin{itemize}

\item \textbf{Delta} 

The Delta of a digital put option is the rate of change of the option price with respect to changes in the underlying asset's price. It can be calculated from the Delta of a digital call option as follows:

$$
\Delta_{DP}(K,T,S_0,V_0) = -\Delta_{DC}(K,T,S_0,V_0) \ .
$$

\item \textbf{Gamma} 

The Gamma of a digital put option is the rate of change in the Delta with respect to changes in the underlying price. This is essentially measuring the second-order sensitivity of the option value to price changes in the underlying asset. It can be calculated from the Gamma of a digital call option as follows:

$$
\Gamma_{DP}(K,T,S_0,V_0) = -\Gamma_{DC}(K,T,S_0,V_0) \ .
$$



\item \textbf{Rho} 

The Rho of a digital call option measures the rate of change of the option price with respect to changes in the risk-free interest rate. It can be represented as follows:
$$
\text{Rho}_{DP}(K,T,S_0,V_0) = 
\frac{\partial DF(0,T)}{\partial r}  - \text{Rho}_{DC}(K,T,S_0,V_0) \ .
$$


\item \textbf{Theta} 

The Theta of a digital put option measures the rate of change of the option price with respect to changes in time, or the time decay of the option. Similary to the call option case, we approximate it using the central difference method. 
$$
 \Theta_{DP}(K,T,S_0,V_0) \approx \frac{DP(K,T-h) - DP(K,T+h)}{2h} \ ,
$$
where $h$ is a small increment in time. 

\item \textbf{Vega} 

In the Heston model, Vega measures the rate of change of the option price with respect to changes in the initial value of the variance $V_0$. For a digital put option, it can be derived as follows:

$$
\text{Vega}_{DP}(K,T,S_0,V_0) = - \text{Vega}_{DC}(K,T,S_0,V_0) \ .
$$






\end{itemize}


\bibliographystyle{unsrt}
\bibliography{../../../DocumentationTemplate/bibliography}
\end{document}


