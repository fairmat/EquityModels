\newcommand{\pluginName}{Variance Gamma Model}
\newcommand{\pluginVersion}{1.0}

\input{../../../DocumentationTemplate/TemplateL3}

\begin{document}
\PluginTitle{\pluginName}{\pluginVersion}

\section{Preface}
\emph{This plug-is has been developed starting from the University of Edinburgh MSC in finance and operation research dissertation of Safe Khampol. The aim of the dissertation was to implement a calibration and Monte Carlo simulation procedure for The Variance Gamma (VG) Model for Matlab and for the Fairmat plug-in system.} 

\section{Introduction}
The Variance Gamma (VG) is a  mathematical model for simulating stock prices and allows closed form 
calculation of European options, one popular financial derivative product. 
The model is both simple and robust at the same time and is a very good 
alternation to the existing models like Heston stochastic volatility model 
and Black-Scholes Model which may be biased when applied to real market 
circumstance. (Comment:Heston is good no comparison with Heston unless)

In the codes, we generated entirely option prices of VG examples by picking 
all the input variables and using VG closed form and VG Fast Fourier 
Transform methods to compute their values. After that, each method that has 
been used in the example will be served as model price in the calibration. 
Then, the VG calibration will estimate three parameters $\theta $, $\sigma $ 
and $\upsilon $ and those parameters will be substituted into computation of 
VG Monte Carlo and Back Scholes option pricing and VG Simulation of Price 
Path as Gamma time-changed Brownian Motion and difference of Gamma.

As a reminder, all the symbols or names in C{\#} and Matlab have the same 
meaning by assuming s$_{0}$ is the current stock price, k is strike price, t 
(or T-t) is time to maturity starting from time 0, q is a dividend, $\theta 
$ (theta) is the drift in the Brownian motion, $\sigma $ (sigma) is the 
volatility of the Brownian motion, and $\upsilon $ (nu) is the variance rate 
of gamma time change. Some methods do not include the dividend, q.

In this work we implemented both valuation models of Black-Scholes and 
Variance Gamma. In VG model, there are various ways to find VG option price. 
However, the recommended VG option pricing models are VG closed form, VG 
Monte Carlo, Fast Fourier Transform methods and will be explained in 
details. In case of the calibration of parameters, either Fast Fourier 
transform or closed-form has different advantages such as time of 
calibration, accuracy of estimated parameters and limitation of constraint. 
Finally, the VG simulation of stock price and comparison of BS and VG option 
are employed by calibrated parameters. These topics will be explained in the 
following sections.

\section[Closed-form formulas for European Call options]
        {Variance Gamma Closed-form formulas for European Call options}

The first formula is closed-form for Variance Gamma European option pricing 
model by Madan, Carr and Eric. The original equation contradicts with the 
following formula because it contains mistakes in the model. The errors were 
in the way of computation of a$_{1}$, b$_{1}$, a$_{2}$, and b$_{2}$.This is 
also confirmed from Professor Peter P. Carr e-mail, but he didn't point out 
where exactly. However, the below formula was implemented by Whitley (2009) 
and I has been re-checked its solution with other Variance Gamma methods.

Its closed-form is originated from the Classic Black-Scholes model and 
adapted to the conditional normality of the process and the Variance Gamma 
process as drift of Brownian motion with time span from a Gamma distribution 
and the risk neutral dynamic of the stock price. As a result, it uses a high 
computation process, due to the modified Bessel function of the second kind 
and integration of the degenerate hyper-geometric function of two variables, 
and is given as

\begin{center}
Call (s$_{0}$, k, t) $=$ s$_{0\, }e^{-rt}\psi $ (a$_{1}$, b$_{1}$, 
$\gamma )$ -- k $e^{-rt} \quad \psi $ (a$_{2}$, b$_{2}$, $\gamma )$
\end{center}

where
\[
\gamma \quad =
\, \frac{t}{\upsilon }
\]
\[
\omega =\, \frac{1}{\upsilon }\, \mathrm{ln?}(1-\, \upsilon \theta -\, 
\frac{1}{2}\sigma^{2}\upsilon )
\]
\[
\zeta =\, \frac{(\ln \left( \frac{s_{t}}{k} \right)+\, \omega t)}{\sigma }
\]
\[
\vartheta =1-\, \upsilon \, (\theta +\, \frac{1}{2}\sigma^{2})
\]
\[
a_{1}\, =\, \zeta \sqrt \frac{\vartheta }{\upsilon } 
\]
\[
b_{1}\, =\, \frac{1}{\sigma }(\theta +\, \sigma^{2})\, \sqrt \frac{\upsilon 
}{\vartheta } 
\]
\[
a_{2}=\zeta \, \sqrt \frac{1}{\upsilon } 
\]
\[
b_{2}=\, \frac{1}{\sigma }\, \theta \sqrt \upsilon 
\]
The function $\psi $ is defined in terms of the modified Bessel function of 
the second kind,${\, K}_{v}(z)$, and the degenerate hyper-geometric function 
of two variables, $\Phi $ has the integral representation Humbert (1920), as
\[
\mathrm{\psi \, }\left( \mathrm{a,\, b,\, \gamma } \right)\mathrm{=\, 
}\frac{c^{\gamma +\frac{1}{2}}e^{sign\left( a \right)c}\left( 1+u 
\right)^{\gamma }}{\sqrt {2\pi } \, ?\left( \gamma \right)\gamma }K_{\gamma 
+\frac{1}{2}}(c)\Phi \left( \gamma ,\, 1-\gamma ,\, 1+\gamma ;\, 
\frac{1+u}{2}\mathrm{,\, -sign}\left( \mathrm{a} \right)\mathrm{c}\, \left( 
\mathrm{1+u} \right) \right)\]\[\mathrm{-sign(a)\, }\frac{c^{\gamma 
+\frac{1}{2}}e^{sign\left( a \right)c}\left( 1+u \right)^{1+\gamma }}{\sqrt 
{2\pi } \, ?\left( \gamma \right)\left( 1+\gamma \right)}\mathrm{\, 
}K_{\gamma -\frac{1}{2}}\left( c \right)\Phi \left( 1+\gamma ,\, 1-\gamma 
,\, 2+\gamma ;\, \frac{1+u}{2},\, -sign\left( a \right)c\left( 1+u \right) 
\right)\]\[+sign(a)\frac{c^{\gamma +\frac{1}{2}}e^{sign\left( a \right)c}\left( 
1+u \right)^{1+\gamma }}{\sqrt {2\pi } ?\left( \gamma \right)\gamma 
}K_{\gamma -\frac{1}{2}}(c)\Phi \left( \gamma ,\, 1-\gamma ,\, 1+\gamma ;\, 
\frac{1+u}{2},\, -sign\left( a \right)c(1+u) \right),
\]
where c $= \quad \left| a \right|\sqrt {2+b^{2}} $, $u=\, \frac{b}{\sqrt 
{2+b^{2}} }$ and
\[
\mathrm{\Phi }\, \left( \alpha ,\, \beta ,\, \gamma ,\, x,\, y \right)=\, 
\frac{?(\gamma )}{?(\alpha )?(\gamma -\alpha )}\, \int_0^1 {u^{\alpha 
-1}{(1-u)}^{\gamma -\alpha -1}{(1-ux)}^{-\beta }e^{uy}\, \partial u} 
\]
where $?(x)$ is the gamma function.


However, the degenerate hyper geometric function of two variables, $\Phi $, 
has a strong singularity. As a result, in Matlab the adaptive Simpson 
quadrature integration will be the most effective method because at least 
there will be solutions for any given parameters. On the other hands, in 
C{\#}, regarding this problem, I adapted Humbert series expansion for 
confluent Hypergeometric function of two variables and wrote a C{\#} for it. 
The Confluent hypergeometric functions of two variables have the following 
integral form:
\[
\Phi (\alpha ,\beta ,\gamma ,x,y)=\frac{\Gamma (\gamma )}{\Gamma (\alpha 
)\Gamma (\gamma -\alpha )}\int\limits_0^1 {v^{\alpha -1}(1-v)^{\gamma 
-\alpha -1}(1-vx)^{-\beta }e^{vy}dv} 
\]
and the series form:
\[
\Phi (\alpha ,\beta ,\gamma ,x,y)=\sum\limits_{m,n=0}^\infty {\frac{(\alpha 
)_{m+n} (\beta )_{m} }{(\gamma )_{m+n} {\mkern 1mu}m!{\mkern 1mu}n!}} 
{\mkern 1mu}x^{m}y^{n} ,\vert x\vert <1,\vert y\vert <1
\]
where the Pochhammer symbol (q)$_{n}$ denotes the rising factorial:
\[
(q)_{n} =\frac{\Gamma (q+n)}{\Gamma (q)}=q{\mkern 1mu}(q+1)\cdots (q+n-1) 
\]

This can imply the series is still convergent with \textbar x\textbar 
\textless 1and without condition \textbar y\textbar \textless 1 because 
\[
x=\frac{1+u}{2}=\frac{1+\frac{b}{\sqrt {2+b^{2}} }}{2}\,\,\,\Rightarrow 
\,\,\vert x\vert \,\le \frac{1}{2}+\frac{\vert b\vert }{2\sqrt {2+b^{2}} 
}=\frac{1}{2}+\frac{\vert b\vert }{2\sqrt {2+b^{2}} }<1,\,\,\,\forall b
\]
means \textbar x\textbar \textless 1 in this model. Formula for my integral:
\[
I(\alpha ,x,y)=\frac{\Gamma (\alpha )}{\Gamma (1+\alpha )}\Phi (\alpha 
,1-\alpha ,1+\alpha ,x,y)=\int\limits_0^1 {v^{\alpha -1}(1-vx)^{\alpha 
-1}e^{vy}dv,\,\,\,0<\alpha ,x,y<1} 
\]

Another way around, it also can use an expansion related to Gauss 
hyper-geometric function ($_{2} F_{1} (a,b;c;x)$ is Gauss hyper-geometric 
function and can be calculated with Matlab function hypergeom), shown as 
below. 
\[
\begin{array}{l}
 \Phi (\alpha ,\beta ,\gamma ;x,y)=\frac{\Gamma (\gamma )}{\Gamma (\alpha 
)\Gamma (\gamma -\alpha )}\int\limits_0^1 {v^{\alpha -1}(1-v)^{\gamma 
-\alpha -1}(1-vx)^{-\beta }e^{vy}dv} \\ 
 =\frac{\Gamma (\gamma )}{\Gamma (\alpha )\Gamma (\gamma -\alpha 
)}\sum\limits_{n=0}^\infty {\frac{y^{n}}{n!}\int\limits_0^1 {v^{n+\alpha 
-1}(1-v)^{\gamma -\alpha -1}(1-vx)^{-\beta }dv} } \\ 
 =\sum\limits_{n=0}^\infty {\frac{\Gamma (\gamma )\Gamma (n+\alpha )}{\Gamma 
(\alpha )\Gamma (n+\gamma )}} \,_{2} \frac{y^{n}}{n!}F_{1} (\beta ,n+\alpha 
;n+\gamma ;x),\,\,\,\,\vert x\vert <1. \\ 
 \end{array}
\]
In conclusion, these formulas should work in theory, but there is an extreme 
result and rarely occurs wrong answers. Both forms produce the same answer.

\section[Monte Carlo Simulation for European Call options]
        {Variance Gamma Monte Carlo Simulation for European Call options}

A Monte Carlo simulation is a procedure for sampling random outcomes for the 
applying process, VG process. It helped to evaluate and analyze investments 
by simulating the various sources if uncertainty affecting their value, and 
then determining their average value of sampling outcomes. Its option 
pricing way is straightforward: sample a sufficiently big number of stock 
price paths of the VG process to obtain VG stock price paths under the risk 
neutral measure between time 0 and t. The initial inputs are time at 
maturity t, skewness $\theta $, volatility $\sigma $, variance rate of the 
gamma process $\upsilon $, risk-free rate r, dividend q, 

, and spot price s0. $\omega $ is the same equation as before,$\, 
\frac{1}{\mathrm{\upsilon }}\ln \left( 1-\theta \mathrm{\upsilon 
}-\frac{\sigma^{2}\mathrm{\upsilon }}{2} \right)$

The sampling g values are generated by using gamma inverse cumulative 
distribution function. The inverse of the gamma cumulative distribution 
function with shape parameters in t / $\upsilon $ and scale parameters in 
$\upsilon $ for the corresponding probabilities in random number from 
uniform distribution of zero mean and unit variance between 0 and 1.
\[
g=gammainv(uniform\left( 0,1 \right)\mathrm{,\, t\, /\, \upsilon },\, 
\mathrm{\upsilon }\, )
\]
Each value of g sample is normally distributed with mean $\theta $g and 
standard deviation$\, \sigma \sqrt g $ with inverse standard normal 
distribution function. Sometimes, g is referred to ``measuring economic time 
or time adjusted for the flow of information'' as well as t is ``the usual 
time measure''.
\[
h=\, \theta g+\, \sigma \sqrt {g\, } norminv(uniform\left( 0,1 \right))
\]
The stock price s$_{t}$ is given by
\[
s_{t}=s_{0}exp\left[ \left( r-q \right)t+\omega t+h \right]
\]

Then, the European call price is
\[
c\left( s_{0};k,t \right)=\, e^{-rt}E\left[ max\left( s_{t}-k,0 \right) 
\right]
\]
Thus, the variance gamma call and put option VG Monte Caro simulations are
\[
c=\, e^{-rt}\frac{\sum\limits_{i=1}^n {max\left[ s_{t}\left( i \right)-k,0 
\right]} }{n}
\]
\[
p=\, e^{-rt}\frac{\sum\limits_{i=1}^n {max\left[ {k-s}_{t}\left( i \right),0 
\right]} }{n}
\]
where n is number of simulation of stock price.

Another way around of characterizing gamma distribution is that the arriving 
rate of information during time t is defined by g. If g is large, large 
arriving information and the sample of stock price is a normal distribution 
in ``h'' from above equation which has a relatively large mean and variance. 
Conversely, if g is small, a small deal of information arrives and the 
sample of stock price has a relatively small mean and variance.

If $\upsilon $ is getting closer to zero in the variance gamma model, the 
value of g will be becoming T with certainty and can be illustrated by using 
the gamma distribution function. In addition, $\omega $ is --$\theta $t from 
using a series expansion for the stock price function; consequently, in the 
limit the distribution of S$_{t}$ has a mean of $S_0 \exp[(r-q)t]$ and a 
standard deviation of $\sigma \sqrt{t}$ then this model converts to 
geometric Brownian motion.

\subsection{Calculation of Variance Gamma Put for European options}

The value of the European put is calculated from the value of the European 
call option by put-call parity, as

\begin{center}
Put (s$_{t}$, k, t) $=$ Call (s$_{t}$, k, t) - s$_{t}$ e$^{-qt} \quad +$ k 
e$^{-rt}$
\end{center}

\section{Calibration of VG parameters from its formula}

Madan, Carr and Chang (The Variance Gamma Process and Option Pricing, 1998) 
also introduced the tuned variance gamma process in the index options market 
and statistical process to the level of the fourth moment. There are varying 
numbers of option prices range and thus, it is typically impossible to 
identify a single set of parameters that would fit right to all options 
prices. As a result, maximum likelihood method is employed to estimate these 
parameters. To be exact, the likelihood employed addresses expected 
heteroskedasticity in option prices for various strikes point by adopting a 
multiplicative error formulation. The maximum likelihood estimation is 
asymptotically equivalent to the minimization of followed, with $\omega_{i}$ as 
the observed market price on the i$^{th}$ and $\hat{\omega}_{i}$ as the 
model price.
\[
k=\sqrt{\frac{1}{M}\sum{i=1}^M[\ln(\omega_i)-ln(\hat{\omega}_{i})]}
\]
These include the average estimated parameter values of the risk neutral 
densities and its accompanying standard deviation throughout the 
observation, and also the minimum and maximum estimated value. After the 
research of model in Matlab, the variance gamma option pricing model, $\hat{\omega}_{i}$, in implementation will be used only two methods: variance 
gamma Closed Form and Fast Fourier transform because they are given a 
reasonable parameters and speed of evaluating computation. In contrast, VG 
Monte Carlo takes a long time to calibrate the parameters.

\section{Simulation of Variance Gamma Price Paths}

Those calibrating three parameters and given parameters will be substituted 
to find variance gamma price path. There are three main methods to simulate 
VG. However, in this paper, two methods are only considered and implemented 
and the third one for simulating VG is approximated by a compound Poisson 
process. The two models based on two representations presented before and 
have closely and correctly distribution. The only advantage of the third 
method that the other two cannot do is that it can be used for any L\'{e}vy 
process. 

In the following figure, the two different algorithms for sequentially 
generating VG sample paths between 0 and T starting from 0 (0 $=$ t$_{0}$ 
\textless t$_{1}$ \textless \textellipsis \textless t$_{N} \quad =$ T) where the 
time increments $\Delta t_i, i=1,...,N$ are given as input as well as VG 
parameters.
\textbf{Figure 1: Simulating VG stock price as Gamma Time-Changed Brownian 
motion}

Input: VG parameters ($\theta $, $\sigma $, $\upsilon )$; time spacing 
$\Delta t_{1},\mathellipsis ,\Delta t_{N}$ subjected to $\sum\limits_{i=1}^N 
{\Delta t_{i}=\, \, T} $

Initialization: Set X$_{0} \quad =$ 0.

Loop from i $=$ 1 to N:

\begin{enumerate}
\item Generate $\Delta G_{i}\, \sim \, ?\left( \frac{\Delta t_{i}}{\mathrm{\upsilon }},\mathrm{\upsilon } \right)\, and\, normal\, Z_{i}\sim N\left( 0,1 \right)$
\end{enumerate}
independent and independent of past random variates

\begin{enumerate}
\item Return $X\left( t_{i} \right)=X\left( t_{i-1} \right)+\, \theta \Delta G_{i}+\sigma \sqrt {\Delta G_{i}} Z_{i}$.
\item Computing stock price: $S\left( t_{i} \right)=S\left( 0 \right)\mathrm{exp?}\left( \left( r+\omega \right)t+X\left( t_{i} \right) \right)$
\end{enumerate}

\textbf{Figure 2: Simulating VG stock price as Difference of Gamma}

Input: VG parameters ($\theta $, $\sigma $, $\upsilon )$; time spacing 
$\Delta t_{1},\mathellipsis ,\Delta t_{N}$ subjected to $\sum\limits_{i=1}^N 
{\Delta t_{i}=\, \, T} $

Initialization: Set X$_{0} \quad =$ 0.

Loop from i $=$ 1 to N:

\begin{enumerate}
\item Generate independent gamma variates $\Delta \gamma_{i}^{-}\sim ?\left( \frac{\Delta t_{i}}{\mathrm{\upsilon }},\, \mathrm{\upsilon }\mu_{n} \right)$, $\Delta \gamma_{i}^{+}\sim ?\left( \frac{\Delta t_{i}}{\mathrm{\upsilon }},\, \mathrm{\upsilon }\mu_{p} \right)$, independently of past random variates
\item Return $X\left( t_{i} \right)=X\left( t_{i-1} \right)+\Delta \gamma_{i}^{+}-\Delta \gamma_{i}^{-}$
\item Computing stock price: $S\left( t_{i} \right)=S\left( 0 \right)\mathrm{exp?}\left( \left( r+\omega \right)t+X\left( t_{i} \right) \right)$
\end{enumerate}

\end{document}
